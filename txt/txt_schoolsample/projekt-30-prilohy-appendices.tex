% Tento soubor nahraďte vlastním souborem s přílohami (nadpisy níže jsou pouze pro příklad)
% This file should be replaced with your file with an appendices (headings below are examples only)

% Umístění obsahu paměťového média do příloh je vhodné konzultovat s vedoucím
% Placing of table of contents of the memory media here should be consulted with a supervisor
%\chapter{Obsah přiloženého paměťového média}

%\chapter{Manuál}

%\chapter{Konfigurační soubor} % Configuration file

%\chapter{RelaxNG Schéma konfiguračního souboru} % Scheme of RelaxNG configuration file

%\chapter{Plakát} % poster

\chapter{Jak pracovat s touto šablonou}
\label{jak}

V této kapitole je uveden popis jednotlivých částí šablony, po kterém následuje stručný návod, jak s touto šablonou pracovat. 

Jedná se o přechodnou verzi šablony. Nová verze bude zveřejněna do~konce roku 2017 a~bude navíc obsahovat nové pokyny ke správnému využití šablony, závazné pokyny k~vypracování bakalářských a diplomových prací (rekapitulace pokynů, které jsou dostupné na~webu) a nezávazná doporučení od vybraných vedoucích, která již teď najdete na~webu (viz odkazy v souboru s literaturou). Jediné soubory, které se v nové verzi změní, budou \texttt{projekt-01-kapitoly-chapters.tex} a \texttt{projekt-30-prilohy-appendices.tex}, jejichž obsah každý student vymaže a nahradí vlastním. Šablonu lze tedy bez problémů využít i~v~současné verzi.

\section*{Popis částí šablony}

Po rozbalení šablony naleznete následující soubory a adresáře:
\begin{DESCRIPTION}
  \item [bib-styles] Styly literatury (viz níže). 
  \item [obrazky-figures] Adresář pro Vaše obrázky. Nyní obsahuje placeholder.pdf (tzv. TODO obrázek, který lze použít jako pomůcku při tvorbě technické zprávy), který se s prací neodevzdává. Název adresáře je vhodné zkrátit, aby byl jen ve zvoleném jazyce.
  \item [template-fig] Obrázky šablony (znak VUT).
  \item [fitthesis.cls] Šablona (definice vzhledu).
  \item [Makefile] Makefile pro překlad, počítání normostran, sbalení apod. (viz níže).
  \item [projekt-01-kapitoly-chapters.tex] Soubor pro Váš text (obsah nahraďte).
  \item [projekt-20-literatura-bibliography.bib] Seznam literatury (viz níže).
  \item [projekt-30-prilohy-appendices.tex] Soubor pro přílohy (obsah nahraďte).
  \item [projekt.tex] Hlavní soubor práce -- definice formálních částí.
\end{DESCRIPTION}

Výchozí styl literatury (czechiso) je od Ing. Martínka, přičemž slovenská a anglická verze (slovakiso a englishiso) jsou jeho překlady s drobnými modifikacemi. Oproti normě jsou v~něm určité odlišnosti, ale na FIT je dlouhodobě akceptován. Alternativně můžete využít styl od Ing. Radima Loskota nebo od Ing. Radka Pyšného\footnote{BP Ing. Radka Pyšného \url{http://www.fit.vutbr.cz/study/DP/BP.php?id=7848}}. Alternativní styly obsahují určitá vylepšení, ale zatím nebyly řádně otestovány větším množstvím uživatelů. Lze je považovat za beta verze pro zájemce, kteří svoji práci chtějí mít dokonalou do detailů a neváhají si nastudovat detaily správného formátování citací, aby si mohli ověřit, že je vysázený výsledek v pořádku.

Makefile kromě překladu do PDF nabízí i další funkce:
\begin{itemize}
  \item přejmenování souborů (viz níže),
  \item počítání normostran,
  \item spuštění vlny pro doplnění nezlomitelných mezer,
  \item sbalení výsledku pro odeslání vedoucímu ke kontrole (zkontrolujte, zda sbalí všechny Vámi přidané soubory, a případně doplňte).
\end{itemize}

Nezapomeňte, že vlna neřeší všechny nezlomitelné mezery. Vždy je třeba manuální kontrola, zda na konci řádku nezůstalo něco nevhodného -- viz Internetová jazyková příručka\footnote{Internetová jazyková příručka \url{http://prirucka.ujc.cas.cz/?id=880}}.

\paragraph {Pozor na číslování stránek!} Pokud má obsah 2 strany a na 2. jsou jen \uv{Přílohy} a~\uv{Seznam příloh} (ale žádná příloha tam není), z nějakého důvodu se posune číslování stránek o 1 (obsah \uv{nesedí}). Stejný efekt má, když je na 2. či 3. stránce obsahu jen \uv{Literatura} a~je možné, že tohoto problému lze dosáhnout i jinak. Řešení je několik (od~úpravy obsahu, přes nastavení počítadla až po sofistikovanější metody). \textbf{Před odevzdáním proto vždy překontrolujte číslování stran!}


\section*{Doporučený postup práce se šablonou}

\begin{enumerate}
  \item \textbf{Zkontrolujte, zda máte aktuální verzi šablony.} Máte-li šablonu z předchozího roku, na stránkách fakulty již může být novější verze šablony s~aktualizovanými informacemi, opravenými chybami apod.
  \item \textbf{Zvolte si jazyk}, ve kterém budete psát svoji technickou zprávu (česky, slovensky nebo anglicky) a svoji volbu konzultujte s vedoucím práce (nebyla-li dohodnuta předem). Pokud Vámi zvoleným jazykem technické zprávy není čeština, nastavte příslušný parametr šablony v souboru projekt.tex (např.: \verb|documentclass[english]{fitthesis}| a přeložte prohlášení a poděkování do~angličtiny či slovenštiny.
  \item \textbf{Přejmenujte soubory.} Po rozbalení je v šabloně soubor \texttt{projekt.tex}. Pokud jej přeložíte, vznikne PDF s technickou zprávou pojmenované \texttt{projekt.pdf}. Když vedoucímu více studentů pošle \texttt{projekt.pdf} ke kontrole, musí je pracně přejmenovávat. Proto je vždy vhodné tento soubor přejmenovat tak, aby obsahoval Váš login a (případně zkrácené) téma práce. Vyhněte se však použití mezer, diakritiky a speciálních znaků. Vhodný název může být např.: \uv{\texttt{xlogin00-Cisteni-a-extrakce-textu.tex}}. K přejmenování můžete využít i přiložený Makefile:
\begin{verbatim}
make rename NAME=xlogin00-Cisteni-a-extrakce-textu
\end{verbatim}
  \item Vyplňte požadované položky v souboru, který byl původně pojmenován \texttt{projekt.tex}, tedy typ, rok (odevzdání), název práce, svoje jméno, ústav (dle zadání), tituly a~jméno vedoucího, abstrakt, klíčová slova a další formální náležitosti.
  \item Nahraďte obsah souborů s kapitolami práce, literaturou a přílohami obsahem svojí technické zprávy. Jednotlivé přílohy či kapitoly práce může být výhodné uložit do~samostatných souborů -- rozhodnete-li se pro toto řešení, je doporučeno zachovat konvenci pro názvy souborů, přičemž za číslem bude následovat název kapitoly. 
  \item Nepotřebujete-li přílohy, zakomentujte příslušnou část v \texttt{projekt.tex} a příslušný soubor vyprázdněte či smažte. Nesnažte se prosím vymyslet nějakou neúčelnou přílohu jen proto, aby daný soubor bylo čím naplnit. Vhodnou přílohou může být obsah přiloženého paměťového média.
  \item Zadání, které si stáhnete v PDF z IS FIT, uložte do souboru \texttt{zadani.pdf} a povolte jeho vložení do práce parametrem šablony v projekt.tex (\verb|documentclass[zadani]{fitthesis}|).
  \item Nechcete-li odkazy tisknout barevně (tedy červený obsah -- bez konzultace s vedoucím nedoporučuji), budete pro tisk vytvářet druhé PDF s tím, že nastavíte parametr šablony pro tisk: (\verb|documentclass[zadani,print]{fitthesis}|).  Barevné logo se nesmí tisknout černobíle!
  \item Vzor desek, do kterých bude práce vyvázána, si vygenerujte v informačním systému fakulty u zadání. Pro disertační práci lze zapnout parametrem v šabloně (více naleznete v souboru fitthesis.cls).
  \item Nezapomeňte, že zdrojové soubory i (obě verze) PDF musíte odevzdat na CD či jiném médiu přiloženém k technické zprávě.
\end{enumerate}

Obsah práce se generuje standardním příkazem \tt \textbackslash tableofcontents \rm (zahrnut v šabloně). Přílohy jsou v něm uvedeny úmyslně.

\subsection*{Pokyny pro oboustranný tisk}
\begin{itemize}
\item \textbf{Oboustranný tisk je doporučeno konzultovat s vedoucím práce.}
\item Je-li práce tištěna oboustranně a její tloušťka je menší než tloušťka desek, nevypadá to dobře.
\item Zapíná se parametrem šablony: \verb|\documentclass[twoside]{fitthesis}|
\item Po vytištění oboustranného listu zkontrolujte, zda je při prosvícení sazební obrazec na obou stranách na stejné pozici. Méně kvalitní tiskárny s duplexní jednotkou mají často posun o 1--3 mm. Toto může být u některých tiskáren řešitelné tak, že vytisknete nejprve liché stránky, pak je dáte do stejného zásobníku a vytisknete sudé.
\item Za titulním listem, obsahem, literaturou, úvodním listem příloh, seznamem příloh a případnými dalšími seznamy je třeba nechat volnou stránku, aby následující část začínala na liché stránce (\textbackslash cleardoublepage).
\item  Konečný výsledek je nutné pečlivě překontrolovat.
\end{itemize}

\subsection*{Styl odstavců}

Odstavce se zarovnávají do bloku a pro jejich formátování existuje více metod. U papírové literatury je častá metoda s~použitím odstavcové zarážky, kdy se u~jednotlivých odstavců textu odsazuje první řádek odstavce asi o~jeden až dva čtverčíky (vždy o~stejnou, předem zvolenou hodnotu), tedy přibližně o~dvě šířky velkého písmene M základního textu. Poslední řádek předchozího odstavce a~první řádek následujícího odstavce se v~takovém případě neoddělují svislou mezerou. Proklad mezi těmito řádky je stejný jako proklad mezi řádky uvnitř odstavce. \cite{fitWeb} Další metodou je odsazení odstavců, které je časté u elektronické sazby textů. První řádek odstavce se při této metodě neodsazuje a mezi odstavce se vkládá vertikální mezera o~velikosti 1/2 řádku. Obě metody lze v kvalifikační práci použít, nicméně často je vhodnější druhá z uvedených metod. Metody není vhodné kombinovat.

Jeden z výše uvedených způsobů je v šabloně nastaven jako výchozí, druhý můžete zvolit parametrem šablony \uv{\tt odsaz\rm }.

\subsection*{Užitečné nástroje}
\label{nastroje}

Následující seznam není výčtem všech využitelných nástrojů. Máte-li vyzkoušený osvědčený nástroj, neváhejte jej využít. Pokud však nevíte, který nástroj si zvolit, můžete zvážit některý z následujících:

\begin{description}
	\item[\href{http://miktex.org/download}{MikTeX}] \LaTeX{} pro Windows -- distribuce s jednoduchou instalací a vynikající automatizací stahování balíčků.
	\item[\href{http://texstudio.sourceforge.net/}{TeXstudio}] Přenositelné opensource GUI pro \LaTeX{}.  Ctrl+klik umožňuje přepínat mezi zdrojovým textem a PDF. Má integrovanou kontrolu pravopisu, zvýraznění syntaxe apod. Pro jeho využití je nejprve potřeba nainstalovat MikTeX.
	\item[\href{http://www.winedt.com/}{WinEdt}] Ve Windows je dobrá kombinace WinEdt + MiKTeX. WinEdt je GUI pro Windows, pro jehož využití je nejprve potřeba nainstalovat \href{http://miktex.org/download}{MikTeX} či \href{http://www.tug.org/texlive/}{TeX Live}. 
	\item[\href{http://kile.sourceforge.net/}{Kile}] Editor pro desktopové prostředí KDE (Linux). Umožňuje živé zobrazení náhledu. Pro jeho využití je potřeba mít nainstalovaný \href{http://www.tug.org/texlive/}{TeX Live} a Okular. 
	\item[\href{http://jabref.sourceforge.net/download.php}{JabRef}] Pěkný a jednoduchý program v Javě pro správu souborů s bibliografií (literaturou). Není potřeba se nic učit -- poskytuje jednoduché okno a formulář pro editaci položek.
	\item[\href{https://inkscape.org/en/download/}{InkScape}] Přenositelný opensource editor vektorové grafiky (SVG i PDF). Vynikající nástroj pro tvorbu obrázků do odborného textu. Jeho ovládnutí je obtížnější, ale výsledky stojí za to.
	\item[\href{https://git-scm.com/}{GIT}] Vynikající pro týmovou spolupráci na projektech, ale může výrazně pomoci i jednomu autorovi. Umožňuje jednoduché verzování, zálohování a přenášení mezi více počítači.
	\item[\href{http://www.overleaf.com/}{Overleaf}] Online nástroj pro \LaTeX{}. Přímo zobrazuje náhled a umožňuje jednoduchou spolupráci (vedoucí může průběžně sledovat psaní práce), vyhledávání ve zdrojovém textu kliknutím do PDF, kontrolu pravopisu apod. Zdarma jej však lze využít pouze s určitými omezeními (někomu stačí na disertaci, jiný na ně může narazit i při psaní bakalářské práce) a pro dlouhé texty je pomalejší.
\end{description}

Pozn.: Overleaf nepoužívá Makefile v šabloně -- aby překlad fungoval, je nutné kliknout pravým tlačítkem na \tt projekt.tex \rm a zvolit \uv{Set as Main File}.


\subsection*{Užitečné balíčky pro \LaTeX}

Studenti při sazbě textu často řeší stejné problémy. Některé z nich lze vyřešit následujícími balíčky pro \LaTeX:

\begin{itemize}
  \item \verb|amsmath| -- rozšířené možnosti sazby rovnic,
  \item \verb|float, afterpage, placeins| -- úprava umístění obrázků,
  \item \verb|fancyvrb, alltt| -- úpravy vlastností prostředí Verbatim, 
  \item \verb|makecell| -- rozšíření možností tabulek,
  \item \verb|pdflscape, rotating| -- natočení stránky o 90 stupňů (pro obrázek či tabulku),
  \item \verb|hyphenat| -- úpravy dělení slov,
  \item \verb|picture, epic, eepic| -- přímé kreslení obrázků.
\end{itemize}

Některé balíčky jsou využity přímo v šabloně (v dolní části souboru fitthesis.cls). Nahlédnutí do jejich dokumentace může být rovněž užitečné.

Sloupec tabulky zarovnaný vlevo s pevnou šířkou je v šabloně definovaný \uv{L} (používá se jako \uv{p}).

